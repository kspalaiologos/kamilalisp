
\chapter{Applied mathematics}

\section{Numerically evaluating rational sums and integrals}

Suppose that we want to compute the values of the following expressions, given some polynomial functions $P(x)$ and $Q(x)$:

$$
\sum_{i=0}^\infty \frac{P(i)}{Q(i)}
$$

$$
\int_0^\infty \frac{P(x)}{Q(x)} dx
$$

\noindent The problem is that these expressions may not converge, or may converge very slowly. Precisely, $\deg P(x) - \deg Q(x) < -1$ is a sufficient condition for divergence. To demonstrate a particular case of performing the summation, consider the following:

$$\sum_{k=1}^\infty \frac{1}{k^2 - a}$$

\noindent where $a$ is a constant. The following equalities hold:

$$\sum_{k=1}^\infty \frac{1}{k^2 - a} = \frac{1-\sqrt{a}\pi\cot(\sqrt{a}\pi)}{2a} = \frac{1}{2\sqrt{a}}\left(\psi\left(1+\sqrt{a}\right)-\psi\left(1-\sqrt{a}\right)\right)$$

\noindent To attempt at understanding why is this happening, recall the following identity due to Euler (1770) for $z \in \mathbb{C} \setminus \mathbb{Z}$:

$$
\pi \cot(\pi z) = z^{-1} + \sum_{v=1}^\infty \frac{2z}{z^2 - v^2}
$$

\noindent A concise proof using Mittag-Leffler's theorem follows. Given $b_n$ as the residues:

$$
f(z) = f(0) + \sum_{n=1}^\infty b_n \left(\frac{1}{z-z_n} + \frac{1}{z_n}\right) = f(0) + \sum_{n=1}^\infty \frac{zb_n}{z_n(z_n-z)}
$$

\noindent Using the contour integral where $C_N$ is a circle enclosing first $N$ poles of $f$:

$$
I_N = \oint_{C_N} \frac{f(\omega)d\omega}{\omega(\omega - z)}
$$

\noindent Using the residue theorem we find:

$$
I_N = -2\pi i \frac{f(0)}{f(z)} + 2\pi i \frac{f(z)}{z} + \sum_{n=1}^N \frac{2\pi i b_n}{z_n(z_n - z)}
$$

\noindent Consider $\cot z - z^{-1}$ to remove a singularity. $b_n$ is found using the residue theorem as follows:

$$
b_n = \lim_{z \to n\pi} (z-n\pi)\cot z = \lim_{z \to n\pi} (z-n\pi) \frac{z \cos z - \sin z}{z \sin z} = 1
$$

\noindent Hence:

$$
\cot z - z^{-1} = \sum_{n=1}^N \frac{1}{z-n\pi} + \frac{1}{n\pi} + \frac{1}{z+n\pi} - \frac{1}{n\pi}
$$

\noindent Substitute and rearrange:

$$
\pi \cot(\pi z) = z^{-1} + \sum_{n=1}^N \left(\frac{1}{z-n} + \frac{1}{z+n}\right) = z^{-1} + \sum_{n=1}^N \frac{2z}{z^2-n^2}
$$

\noindent Knowing where the first part of the equality comes from, the next logical step is reasoning about the second equality:

$$
\frac{1-\sqrt{a}\pi\cot(\sqrt{a}\pi)}{2a} = \frac{1}{2\sqrt{a}}\left(\psi\left(1+\sqrt{a}\right)-\psi\left(1-\sqrt{a}\right)\right)
$$

\noindent Use the recurrence relation of the digamma function once:

$$
\frac{1-\sqrt{a}\pi\cot(\sqrt{a}\pi)}{2a} = \frac{1}{2\sqrt{a}}\left(\frac{1}{\sqrt{a}}+\psi\left(\sqrt{a}\right)-\psi\left(1-\sqrt{a}\right)\right)
$$

\noindent Now it is possible to apply the digamma reflection formula given as follows:

$$
\psi (1-x)-\psi (x)=\pi \cot \pi x
$$

\noindent The problem hence reduces to:

$$
\frac{1}{2\sqrt{a}}\left(\frac{1}{\sqrt{a}}-\pi\cot\pi\sqrt{a}\right) = \frac{1}{2\sqrt{a}}\left(\frac{1-\sqrt{a}\pi\cot\pi\sqrt{a}}{\sqrt{a}}\right) = \frac{1-\sqrt{a}\pi\cot\pi\sqrt{a}}{2a}
$$

The general algorithm for convergent series implemented below is as follows. Denote the roots of the polynomial $Q(x+1)$ (obtained using the most convenient means; counting multiplicity) as $R_0 \dots R_m$. The following equality holds if $P(x)=1$:

$$
\sum_{n=1}^\infty \frac{1}{Q(n)} = -\sum_{\omega \in R} \frac{\psi(-\omega)}{Q'(\omega + 1)}
$$

If $P(x) \ne 1$ (i.e. the rational function is not an inverse of some polynomial), the formula is as follows:

$$
\sum_{n=1}^\infty \frac{P(n)}{Q(n)} = -\sum_{\omega \in R} \frac{P(\omega + 1) \psi(-\omega)}{Q'(\omega + 1)}
$$

To start, define a few helper functions - evaluating a polynomial at a point, computing the symbolic derivative of a polynomial and substituting $x+1$ into the polynomial to transform the \textit{litte-endian} coefficient list:

\begin{Verbatim}
    (def polyevl
      [(inner-product cmplx64:+ cmplx64:*) #0 ^cmplx64:**&[#0 range@tally]])

    (def polyderv cdr@[* #0 range@tally])

    (defun poly+1 (c)
      (let-seq
        (def bin-mat (:[:^binomial #0 \range $(+ 1)]@range@tally c))
        (def coeff-mat (:$(take (tally c)) (* c bin-mat)))
        (:$(foldl1 +)@transpose coeff-mat)))
\end{Verbatim}

The function is trivially defined as follows:

\begin{Verbatim}
    (defun rsum (p q) (let-seq
      (def q+1 (poly+1 q))
      (def p+1 $(polyevl (poly+1 p)))
      (def q+1p $(polyevl (polyderv q+1)))
      (def R (cmplx64:solve q+1))
      (def V [/ [* p+1 cmplx64:digamma@-] q+1p])
      (cmplx64:neg (foldl + 0 (:V R)))))
\end{Verbatim}

